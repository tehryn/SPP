\documentclass[10pt,xcolor=pdflatex]{beamer}
\usepackage{newcent}
\usepackage[utf8]{inputenc}
\usepackage[czech]{babel}
\usepackage{hyperref}
\usepackage{fancyvrb}
\usepackage{hyperref}
\usepackage[font={small,it}]{caption}
\usepackage{listings}
%\usetheme[]{FIT}
\usetheme[CZlogo]{FIT} % CZ logo

\lstdefinestyle{customc}{
  belowcaptionskip=1\baselineskip,
  breaklines=true,
  frame=single,
  xleftmargin=\parindent,
  language=C,
  showstringspaces=false,
  basicstyle=\footnotesize\ttfamily,
  keywordstyle=\bfseries\color{green!40!black},
  commentstyle=\itshape\color{purple!40!black},
  identifierstyle=\color{blue},
  stringstyle=\color{orange},
}

\lstdefinestyle{customasm}{
  belowcaptionskip=1\baselineskip,
  frame=L,
  xleftmargin=\parindent,
  language=[x86masm]Assembler,
  basicstyle=\footnotesize\ttfamily,
  commentstyle=\itshape\color{purple!40!black},
}

\lstset{escapechar=@,style=customc}

%%%%%%%%%%%%%%%%%%%%%%%%%%%%%%%%%%%%%%%%%%%%%%%%%%%%%%%%%%%%%%%%%%

\usepackage{filecontents}

\title[Optimalizace softwarového zpracování CRC]{Optimalizace softwarového zpracování CRC}

\author[]{Jiří Matějka}

% CZ verzia
\institute[]{Vysoké učení technické v Brně, Fakulta informačních technologií\\
Božetěchova 1/2 612 66 Brno\\
xmatej52@stud.fit.vutbr.cz}


%\date{14. 6. 2018}
\date{\today}\usepackage[font={small,it}]{caption}
%\date{} % bez data

%%%%%%%%%%%%%%%%%%%%%%%%%%%%%%%%%%%%%%%%%%%%%%%%%%%%%%%%%%%%%%%%%%

\begin{document}

\frame[plain]{\titlepage}
\begin{frame}
  \frametitle{Cyklický redundantní součet}
  \begin{itemize}
      \item Jeden z nejčastěji používaných algoritmů pro detekci chyb přenosu dat
      \item Hardwarové i softwarové implementace
      \item Slabá pro odhalení záměrné změny dat (není schopen odhalit všechny chyby)
      \item Při náhodné změně vstupní posloupnosti roste pravděpodobnost odhalení chyby spolu s šířkou klíče (dělitele)
      \item Používá se ve spoustě komunikačních protokolů (Ethernet, asynchronous transfer mode, ...)
  \end{itemize}
\end{frame}

\begin{frame}
  \frametitle{Princip výpočtu odesílané zprávy}
  Nechť $M$ je zpráva, kterou je třeba odeslat. $C$ je dělitel o délce $c$ bitů,
  kde první a poslední bit má hodnotu 1. Potom algoritmus výpočtu odesílané zprávy
  bude následující:
  \begin{enumerate}
      \item Na konec zprávy M přidej $c - 1$ nul.
      \item Výsledek vyděl dělitelem $C$ ve zbytkové třídě $2$
      \item Zbytkem po dělení označme $R$, které je $c - 1$ dlouhé
      \item Na konec zprávy M přidej $R$ (namísto přidaných nul) a výsledek odešli příjemci
  \end{enumerate}
  
  Nechť $M'$ je zpráva, kterou obdrží příjemce. \\
  Pokud $Y \mod C = 0$, potom
  příjemce předpokládá, že přijatá zpráva neobsahuje chyby (ne vždy to ale musí být pravda).
\end{frame}

\begin{frame}
    \frametitle{Hardwarová implementace}
    \begin{itemize}
        \item Založena na posuvných registrech a XOR operacích
        \item Polynom je reprezentován sekvencí binárních hodnot v registrech
    \end{itemize}
    \begin{figure}
        \includegraphics[width=\linewidth]{img/crc16_hw.png}
        \caption{ 16 bitová CRC s využitím LFSR (Posuvný registr s lineární zpětnou vazbou) }
        \caption{ Zdroj: \url{https://www.intel.com/content/dam/www/programmable/us/en/pdfs/literature/an/an049_01.pdf} }
    \end{figure}
\end{frame}

\begin{frame}
    \frametitle{Softwarové implementace}
    Algoritmus založený na principu posuvného registru:
    \begin{itemize}
        \item Jednoduché řešení
        \item Opakuje cyklus pro každý bit dat
        \item Pomalé
    \end{itemize}
    Algoritmus s předvypočítanou tabulkou (Table-driven CRC Calculation):
    \begin{itemize}
        \item O něco složitější řešení
        \item Před prvním výpočtem je třeba inicializovat tabulku
        \item Opakuje cyklus pro každý byte dat
        \item Mnohem rychlejší, ale stále pomalejší než HW řešení
    \end{itemize}
\end{frame}

\begin{frame}
    \frametitle{Optimalizace výpočtu - CRC4}
    Pro jednoduchost budeme pracovat s crc4 a polynomem $C = x^4 + x + 1$.
    \begin{figure}
        \includegraphics[width=0.66 \linewidth]{img/crc4_hw.png}
        \\ Zdroj: \url{https://ieeexplore.ieee.org/document/6987839}
    \end{figure}
\end{frame}

\begin{frame}
    \frametitle{Sekvenčí algoritmus CRC4}
    Algoritmus implementující následující CRC4 s využitím tabulky bude vypadat
    následovně:
    \lstinputlisting{src/crc4.c}
\end{frame}

\begin{frame}
    \frametitle{Vzorce}
    Rozdělme si zprávu na jednotlivé bity. Každý bit označme jako $d_n$ kde $n$ je pořadí bitu.
    Hodnotu v jednotlivých registrech označme jako $x_{n,t}$, kde $n$ je pořadí registru a
    $t$ je pořadí prováděného cyklu. Výsledek každého cyklu může být potom reprezentován jako:

    \begin{align}
        x_{0,t+1}&=x_{1,t} \nonumber \\
        x_{1,t+1}&=x_{2,t} \nonumber \\
        x_{2,t+1}&=x_{3,t} \oplus x_{0,t}\oplus d_{t} \nonumber \\
        x_{3,t+1}&=x_{0,t} \oplus d_{t} \nonumber
    \end{align}
\end{frame}

\begin{frame}
    \frametitle{Výsledeky jednotlivých cyklů}
    \begin{figure}
        \includegraphics[width=0.66 \linewidth]{img/crc4_cycle_result.png}
        \\ Zdroj: \url{https://ieeexplore.ieee.org/document/6987839}
    \end{figure}
\end{frame}

\begin{frame}
    \frametitle{Clock 3 - výsledek}
    Poslední řádek tabulky, Clock 3, reprezentuje výsledek.
    Nechť $y_n=x_{n,0} \oplus d_n$, potom nám po dosazení do
    Clock 3 vyjde:
    \begin{align}
        x_{0,4}&=y_0 \oplus y_1 \nonumber \\
        x_{1,4}&=y_1 \oplus y_2 \nonumber \\
        x_{2,4}&=y_3 \oplus y_2 \oplus y_0 \nonumber \\
        x_{3,4}&=y_3 \oplus y_0 \nonumber
    \end{align}
\end{frame}

\begin{frame}
    \frametitle{Optimalizace}
    V softwarové implementaci CRC známe všechny bity dat ($d_{0..3}$) a známe
    všechny bity akumulátoru ($d_{0..3,0}$). Můžeme tedy položit
    před samotným výpočtem programu $x = x \oplus d$. Vznikne následující
    implementace algoritmu:
    \lstinputlisting{src/crc4_opt1.c}
\end{frame}

\begin{frame}
    \frametitle{Výsledky}
    Čeho bylo touto modifikací dosaženo?
    \begin{itemize}
        \item Optimalizace algoritmu využívající tabulku
        \item Na 32 bitovém procesoru zpracovávo najednou 32 bitů dat
        \item Maximální délka dělícího polynomu je 32 bitů
        \item Ušetření 3 instrukcí za každých 32 bitů dat
    \end{itemize}
\end{frame}

\begin{frame}
    \frametitle{Další experimentování}
    Rovnici přepíšeme do tabulky:
    \begin{table}[]
        \begin{tabular}{lllll}
            $y_3$ & $y_3$ & $y_1$ & $y_0$ &  \\
            $y_0$ & $y_2$ & $y_2$ & $y_1$ &  \\
                  & $y_0$ &       &       &  \\
        \end{tabular}
    \end{table}
    Ve sloupcích popřeházíme některé řádky:
    \begin{table}[]
        \begin{tabular}{lllll}
            $y_3$ & $y_2$ & $y_1$ & $y_0$ &  \\
            $y_0$ &       &       &       &  \\
                  & $y_3$ & $y_2$ & $y_1$ &  \\
                  & $y_0$ &       &       &  
        \end{tabular}
    \end{table}
    Nechť $z_n = y_n \oplus y_{n-3} = y_n \oplus (y_n << 3)$:
    \begin{table}[]
        \begin{tabular}{lllll}
            $z_3$ & $z_2$ & $z_1$ & $z_0$ &  \\
                  & $z_3$ & $z_2$ & $z_1$ &
        \end{tabular}
    \end{table}
    Výsledkem potom bude $z =z \oplus (z >> 1)$
\end{frame}
\begin{frame}[fragile]
    \frametitle{Efektivní algoritmus bez využtí tabulky}
    Výše uvedený algoritmus je tak jednoduchý, že ho lze přepsat na pár řádků v assembleru.
\begin{verbatim}
// https://ieeexplore.ieee.org/document/6987839
eor rO, r1          // y = x ^ d
eor rO, rO, lsl #3  // z = y ^ ( y << 3 )
and rO, #OxF        // z = z & 0xF
eor rO, rO, lsr #1  // crc = z ^ ( z >> 1 )
\end{verbatim}
\end{frame}

\begin{frame}
    \frametitle{Čeho bylo dosaženu}
    \begin{itemize}
        \item Algoritmus pro CRC4 byl výrazně zoptimalizován
        \item Vzorce pro delší šířku klíčů jsou generováný programy v Lispu
        \begin{itemize}
              \item Model úspěšně aplikován na Ethernet CRC--32
              \item V některých dalších případech chybí buď výpočetní výkon nebo metoda není efektivnější než algoritmus s využitím tabulky
        \end{itemize}
    \end{itemize}
\end{frame}

\begin{frame}
    \frametitle{Porovnání jednotlivých implementací}
    Uvedené hodnoty byly naměřeny na stejném stroji a reprezentují
    počet nanosekund potřebných na zpracování jednoho bytu.
    Algoritmy byly implementovány v C\texttt{++} a spuštěny na
    Samsung Chromebook s $1.7$ GHz ARM Cortex--A15 CPU s OS Chrubuntu $12.04$ Linux.
    Každá hodnota byla měřena $100\times$ a byl vybrán vždy ten nejkratší čas. Pro výpočet
    CRC byla použita náhodně generovaná zpráva o délce 65536 bytů.
    \begin{table}[]
        \begin{tabular}{llllll}
                  & Serial  & Table1 & Table2 & Parralel & \\
            CRC16 & $22.35$ & $5.29$ & $4.85$ & $2.06$   & \\
            CRC32 & $21.55$ & $5.29$ & $4.85$ & $3.16$   &
        \end{tabular}
    \end{table}
\end{frame}

\begin{frame}
    \frametitle{Literatura}
    \begin{itemize}
        \item J. R. Engdahl and D. Chung, Fast parallel CRC implementation in software [online], Soul: Automation and Systems (ICCAS 2014). 2014, Dostupné z \url{https://ieeexplore.ieee.org/document/6987839 }
        \item Altera Corporation, Implementing CRCCs in Altera Devices [online], San Francisko: Altera Corporation. 2005, Dostupné z: \url{https://www.intel.com/content/dam/www/programmable/us/en/pdfs/literature/an/an049_01.pdf}
        \item Norman Matloff, Cyclic Redundancy Checking [online], University of California: University of Californi -- department of Computer Science, dostupné z: \url{http://heather.cs.ucdavis.edu/~matloff/Networks/CRC/Old/ErrChkCorr.NEW.tex}
        
    \end{itemize}
\end{frame}

\begin{frame}
    \frametitle{Bonus 1}
    \lstinputlisting{src/crc16_no_table.c}
\end{frame}

\begin{frame}
    \frametitle{Bonus 2}
    \lstinputlisting{src/crc16.c}
\end{frame}

\end{document}
